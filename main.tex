% 使用ctexart文档类(用XeLaTeX编译,直接支持中文)
\documentclass{ctexart}

%导言区,可以在此引入必要的宏包
\usepackage{boxie}
\usepackage{hyperref}

\usepackage{graphicx}

% ========设置标题的格式========
\ctexset{
  section = {
    format+ = \zihao{-4} \heiti \raggedright,
    name = {,、},
    number = \chinese{section},
    beforeskip = 1.0ex plus 0.2ex minus .2ex,
    afterskip = 1.0ex plus 0.2ex minus .2ex,
    aftername = \hspace{0pt}
  },
  subsection = {
    format+ = \zihao{5} \heiti \raggedright,
    % name={\thesubsection、},
    name = {,、},
    number = \arabic{subsection},
    beforeskip = 1.0ex plus 0.2ex minus .2ex,
    afterskip = 1.0ex plus 0.2ex minus .2ex,
    aftername = \hspace{0pt}
  }
}

\newcommand{\qtmark}[1]{``#1''}

\title{\Large \heiti 采用tcolorbox宏包设计的用于排版终端窗口和代码的宏
  包boxie.sty}
\author{\zihao{4} \fangsong 耿楠\\\small \songti 西北农林科技大学信息
  工程学院,陕西$\cdot$杨凌,712100}

\begin{document} %在document环境中撰写文档

\maketitle

\begin{abstract}
  在使用\url{https://github.com/latexstudio/ChenLaTeXBookTemplate}提供
  的\LaTeX 书籍模板排版时,发现该模板中设计代码盒子环境和命令非常实用,
  另外,该模板还提供了排版终端命令窗口的基本思路。基于此,通过研读该模
  板的代码,重新设计了用于个Ubuntu、Windows和Mac终端命令窗口的12个排版
  环境和12个从文件读取窗口内容的排版命令,以期更为方便的排版终端命令窗
  口。

  该宏包可以为经常有编写终端命令窗口和代码排版的人员提供帮助,但由于作
  者水平有限,一定存在不足之处,欢迎大家多提宝贵意见和建议。
\end{abstract}

\section{终端窗口盒子环境与命令}
\subsection{终端窗口分类}
\begin{itemize}
\item 带底部说明的黑底白字(用于屏幕阅读)
\item 带底部说明的白底黑字(用于打印输出)
\item 无底部说明的黑底白字(用于屏幕阅读)
\item 无底部说明的白底黑字(用于打印输出)
\end{itemize}

\subsection{环境与命令命名规范}

\begin{itemize}
\item 操作系统前缀
  \begin{itemize}
  \item ubt:Ubuntu
  \item win:Windows
  \item mac:Mac OS
  \end{itemize}
\item 配色中缀
  \begin{itemize}
  \item dark:黑底白字
  \item light:白底黑字
  \end{itemize}
\item 底部说明中缀
  \begin{itemize}
  \item c:有底部说明
  \item 空:无底部说明
  \end{itemize}
\item 文件后缀
  \begin{itemize}
  \item file:窗口内容来自文件
  \item 空:窗口内容在环境中
  \end{itemize}
\end{itemize}
\subsection{环境与命令列表}
共计12个环境,分别是:
\begin{itemize}
\item Ubuntu
  \begin{itemize}
  \item \qtmark{ubtdarkc}:带底部说明的黑底白字
  \item \qtmark{ubtlightc}:带底部说明的白底黑字
  \item \qtmark{ubtdark}:无底部说明的黑底白字
  \item \qtmark{ubtlight}:无底部说明的白底黑字
  \end{itemize}
\item Windows
  \begin{itemize}
  \item \qtmark{windarkc}:带底部说明的黑底白字
  \item \qtmark{winlightc}:带底部说明的白底黑字
  \item \qtmark{windark}:无底部说明的黑底白字
  \item \qtmark{winlight}:无底部说明的白底黑字
  \end{itemize}
\item Mac
  \begin{itemize}
  \item \qtmark{macdarkc}:带底部说明的黑底白字
  \item \qtmark{maclightc}:带底部说明的白底黑字
  \item \qtmark{macdark}:无底部说明的黑底白字
  \item \qtmark{maclight}:无底部说明的白底黑字
  \end{itemize}
\end{itemize}

共计12个命令,分别是:
\begin{itemize}
\item Ubuntu
  \begin{itemize}
  \item \qtmark{\textbackslash ubtdarkcfile}:带底部说明的黑底白字,内
    容来自文件
  \item \qtmark{\textbackslash ubtlightcfile}:带底部说明的白底黑字,内
    容来自文件
  \item \qtmark{\textbackslash ubtdarkfile}:无底部说明的黑底白字,内容
    来自文件
  \item \qtmark{\textbackslash ubtlightfile}:无底部说明的白底黑字,内
    容来自文件
  \end{itemize}
\item Windows
  \begin{itemize}
  \item \qtmark{\textbackslash windarkcfile}:带底部说明的黑底白字,内
    容来自文件
  \item \qtmark{\textbackslash winlightcfile}:带底部说明的白底黑字,内
    容来自文件
  \item \qtmark{\textbackslash windarkfile}:无底部说明的黑底白字,内容
    来自文件
  \item \qtmark{\textbackslash winlightfile}:无底部说明的白底黑字,内
    容来自文件
  \end{itemize}
\item Mac
  \begin{itemize}
  \item \qtmark{\textbackslash macdarkcfile}:带底部说明的黑底白字,内
    容来自文件
  \item \qtmark{\textbackslash maclightcfile}:带底部说明的白底黑字,内
    容来自文件
  \item \qtmark{\textbackslash macdarkfile}:无底部说明的黑底白字,内
    容来自文件
  \item \qtmark{\textbackslash maclightfile}:无底部说明的白底黑字,内
    容来自文件
  \end{itemize}
\end{itemize}
\subsection{基本使用语法}
以黑底白字Ubuntu终端窗口排版为例,其排版语法为:
\subsubsection{环境}
\begin{langPyOne}[tex]{带底部说明的黑底白字环境}
  \begin{ubtdarkc}{底部说明}{标题}
    窗口内容
  \end{ubtdarkc}
\end{langPyOne}
\begin{langPyOne}[tex]{无底部说明的黑底白字环境}
  \begin{ubtdark}{标题}
    窗口内容
  \end{ubtdark}
\end{langPyOne}
\subsubsection{命令}
\begin{langPyOne}[tex]{带底部说明的黑底白字环境}
  \ubtdarkcfile{底部说明}{标题}{窗口内容文件名}
\end{langPyOne}
\begin{langPyOne}[tex]{无底部说明的黑底白字环境}
  \ubtdarkfile{标题}{窗口内容文件名}
\end{langPyOne}

\subsection{排版示例}
\subsection{Ubuntu终端窗口}
% % 用环境排版,内容直接录入
% \begin{ubtdark}{xxxxxx@xxxxxx-lap:~}
%   xxxxxx@xxxxxx-lap:~$ ls
%   Desktop     Downloads   p2         Public   Templates   Videos
%   Documents   Music       Pictures   snap    '#test#'     workspace
%   xxxxxx@xxxxxx-lap:~$
% \end{ubtdark}
% \begin{ubtlight}{xxxxxx@xxxxxx-lap:~}
%   xxxxxx@xxxxxx-lap:~$ ls
%   Desktop     Downloads   p2         Public   Templates   Videos
%   Documents   Music       Pictures   snap    '#test#'     workspace
%   xxxxxx@xxxxxx-lap:~$
% \end{ubtlight}

% % 用命令排版,内容来自文件
% \ubtdarkcfile{ls 是查看文件命令}{xxxxxx@xxxxxx-lap:~}{testls.sh}
% \ubtlightcfile{ls 是查看文件命令}{xxxxxx@xxxxxx-lap:~}{testls.sh}    
% \subsection{Windows命令行窗口}
% % 用命令排版,内容来自文件
% \windarkfile{显示当前目录中的文件}{test.bat}
% \winlightfile{显示当前目录中的文件}{test.bat}

% % 用环境排版,内容直接录入
% \begin{windarkc}{dir 是查看文件命令}{显示当前目录中的文件}
%   C:\Users\Administrator 的目录
%   2018/07/21  15:39    <DIR>          .
%   2018/07/21  15:39    <DIR>          ..
%   2018/07/13  08:35    <DIR>          3D Objects
%   2018/07/13  08:35    <DIR>          Contacts
%   2018/08/06  07:46    <DIR>          Desktop
%                   2 个文件      6,029,312 字节
%                  18 个目录 14,035,107,840 可用字节 
% \end{windarkc}
% \begin{winlightc}{dir 是查看文件命令}{显示当前目录中的文件}
%   C:\Users\Administrator 的目录
%   2018/07/21  15:39    <DIR>          .
%   2018/07/21  15:39    <DIR>          ..
%   2018/07/13  08:35    <DIR>          3D Objects
%   2018/07/13  08:35    <DIR>          Contacts
%   2018/08/06  07:46    <DIR>          Desktop
%                   2 个文件      6,029,312 字节
%                  18 个目录 14,035,107,840 可用字节     
% \end{winlightc}
% \subsection{Mac终端窗口}
% % 用命令排版,内容来自文件
% \macdarkfile{xxxxxx@lap:~}{testsh}
% \maclightfile{xxxxxx@lap:~}{testsh}
% \macdarkcfile{克隆Github远程仓库}{xxxxxx@lap:~}{testsh}
% \maclightcfile{克隆Github远程仓库}{xxxxxx@lap:~}{testsh}

\subsection{通用环境与命令}
定义了2个通用Ubuntu黑底白字样式的终端窗口环境,可以通过可选参数指定窗口内容的语
言,其基本语法是
\begin{langPyOne}[tex]{有底部说明}
  \begin{GitExample}[代码语言]{底部说明}{标题}
     ...
  \end{GitExample} 
\end{langPyOne}
和
\begin{langPyOne}[tex]{无底部说明}
  \begin{GitExampla}[代码语言]{标题}
     ...
  \end{GitExampla} 
\end{langPyOne}
及
\begin{langPyOne}[tex]{无底部说明}
  \gitfile[代码语言]{标题}{文件名}
\end{langPyOne}  

\section{代码排版环境与命令}
\subsection{代码排版样式分类 }
\begin{itemize}
\item 带底部说明无交叉引用
\item 无底部说明无交叉引用
\item 无底部说明有交叉引用
\end{itemize}
\subsection{环境与命令列表}
\subsubsection{环境}
\begin{itemize}
\item \qtmark{langPyTwo}:带底部说明无交叉引用代码排版环境
\item \qtmark{langPyOne}:无底部说明无交叉引用代码排版环境
\item \qtmark{langCVOne}:无底部说明有交叉引用代码排版环境
\end{itemize}
\subsubsection{命令}
\begin{itemize}
\item \qtmark{\textbackslash langPyfile}:无底部说明无交叉引用代码排版
  命令,代码带自文件
\item \qtmark{\textbackslash langCVfile}:无底部说明有交叉引用代码排版
  命令,代码带自文件
\end{itemize}

\subsection{基本语法}
\begin{langPyOne}[tex]{有底部说明无交叉引用代码排版环境}
  \begin{langPyTwo}[语言]{底部说明}{标题}
     ...
   \end{langPyTwo} 
\end{langPyOne}

\langPyfile[tex]{无底部说明无交叉引用代码排版环境}{langpyone.tex} 

\begin{langPyOne}[tex]{无底部说明有交叉引用代码排版环境}
  \begin{langCVOne}[语言][交叉引用标签][显示语言名]{标题}
     ...
   \end{langCVOne} 
\end{langPyOne}

\begin{langPyOne}[tex]{排版命令使用}
\langCVfile[语言][交叉引用标签][语言名显示]{标题}{文件名}
\langPyfile[语言]{标题}{文件名}
\end{langPyOne}
其中, 方括号的参数是可选的。\qtmark{[语言名显示]}是显示在标题中要排版
代码的语言名称。

\subsection{代码排版实例}
\begin{langPyTwo}[c]{这是C语言代码排版一个实例}{C语言代码排版
    (langPyTwo环境)}
  #include<stdio.h>
  #include<stdlib.h>

  int main()
  {
    printf("Hello World!\n");
    return 0;
  }
\end{langPyTwo}

\begin{langPyOne}[java]{Java语言代码排版(langPyOne环境)}
  public class HelloWorld {
    public static void main(String[] args){
      System.out.println("Hello World!");
    }
  }
\end{langPyOne}

\langPyfile[c]{C语言代码排版(langPyfile命令)}{testc.c}

用\qtmark{langCVOne}环境可以实现交叉引用,如代码\ref{ccode01}所示。
\begin{langCVOne}[python][ccode01][Python]{Python(langCVOne环境)}
  # -*- coding: UTF-8 -*-

  # Filename : helloworld.py
  # 该实例输出 Hello World!

  print('Hello World!')
\end{langCVOne}

用\qtmark{\textbackslash langCVfile}命令可以实现交叉引用,如代码\ref{ccode02}所示。
\langCVfile[matlab][ccode02][Matlab]{Matlab(langCVfile命令)}{testm.m}


\section{注意事项}
\subsection{字体}
本宏包需要使用fontawesome5图标字体支持,请在
\url{https://fontawesome.com/}下载安装。
\subsection{代码排版引擎}
本宏包建议使用\qtmark{minted}宏包实现代码的排版,用\qtmark{xelatex
  --shell-escape main.tex}编译tex文件,但如果没有安装minted
需要的python及其pygments模块,请提前安装该模块。

若在编译是不使用\qtmark{--shell-escape}参数,则会自动切换到用listings
排版代码,注意有部分代码名称与pygments定义不一致,请自行查阅相关手册。

\end{document}

%%% Local Variables:
%%% mode: latex
%%% TeX-master: t
%%% End:
